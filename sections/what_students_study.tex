\subsection{Что изучают студенты}

\begin{frame}
  \frametitle{Что изучают студенты}
  \framesubtitle{Пример с занятия по математическуому анализу.}
  \begin{thm} 
    Не существует наибольшего простого числа.
  \end{thm}
  \begin{proof}
    \begin{enumerate}
      \item<1-| alert@1> Предположим, что $p$ - самое большое простое число.
      \item<2-> Пусть $q$ - произведение первых $p$ чисел.
      \item<3-> Тогда $q+1$ не делится ни на один из них.
      \item<1-> Таким образом, $q +1$ также является простым числом и больше, чем $p$.\qedhere
    \end{enumerate}
  \end{proof}
\end{frame}
