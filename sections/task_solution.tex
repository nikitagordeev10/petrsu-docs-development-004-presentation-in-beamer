\begin{frame}[label=task_solution]
\frametitle{Решение задачи }
\tiny
Решение:
Пусть $r$ - радиус основания конуса, а $h$ - высота конуса. Тогда объем конуса может быть выражен как $V=\frac{1}{3}\pi r^2 h$. Чтобы максимизировать объем, мы должны найти максимальную возможную высоту и радиус конуса. Мы можем использовать геометрический способ, чтобы решить эту задачу. Рассмотрим сечение куба, проходящее через две смежные вершины и основание конуса.

Из подобия треугольников можно получить, что $r=\frac{h}{3}$. Кроме того, как мы видим на рисунке, $h+2r=12$, поэтому $h=12-2r$. Подставляя это значение $h$ в формулу для объема, получаем:

$$V=\frac{1}{3}\pi r^2 (12-2r)$$

Теперь мы можем найти максимальный объем, найдя максимум этой функции. Для этого возьмем производную и приравняем ее к нулю:

$$\frac{\mathrm{d} V}{\mathrm{d} r} = 4\pi r - 2\pi r^2 = 0$$

Отсюда получаем, что $r=2$ см, а значит $h=12-2r=8$ см. Значение радиуса, найденное таким образом, является максимальным, поэтому максимальный объем конуса равен:

$$V=\frac{1}{3}\pi \cdot 2^2 \cdot 8 = \frac{32}{3}\pi \approx 33.51 \text{ см}^3$$

Ответ: максимальный объем конуса, который можно вырезать из куба со стороной $12$ см, равен $\frac{32}{3}\pi$.

\hyperlink{example_solution_back}{\beamerbutton{Вернуться}}
\end{frame}